% Created 2021-08-25 Wed 10:47
% Intended LaTeX compiler: pdflatex
\documentclass[bigger]{beamer}
\usepackage[utf8]{inputenc}
\usepackage[T1]{fontenc}
\usepackage{graphicx}
\usepackage{grffile}
\usepackage{longtable}
\usepackage{wrapfig}
\usepackage{rotating}
\usepackage[normalem]{ulem}
\usepackage{amsmath}
\usepackage{textcomp}
\usepackage{amssymb}
\usepackage{capt-of}
\usepackage{hyperref}
\usetheme[progressbar=foot]{metropolis}
\author{Edmund Miller}
\date{2021-04-22 Thu}
\title{Cell-cycle-gated feedback control mediates desensitization to interferon}
\hypersetup{
 pdfauthor={Edmund Miller},
 pdftitle={Cell-cycle-gated feedback control mediates desensitization to interferon},
 pdfkeywords={\href{../../concepts/biology/sequencing/chip\_seq.org}{ChIP-seq}},
 pdfsubject={a framework for deterministic machine learning},
 pdfcreator={Emacs 28.0.50 (Org mode 9.5)}, 
 pdflang={English}}
\begin{document}

\maketitle

\section{Introduction}
\label{sec:org390ef08}
\begin{frame}[label={sec:org06f496f}]{Hypothesis}
\begin{itemize}
\item Does pretreatment with IFN-\(\alpha\) result in desensitization?
\item Can a computational model be created to predict IFN responses?
\item What role does the cell cycle play in the IFN response?
\end{itemize}
\end{frame}

\section{Results}
\label{sec:org0e5db03}

\section{IFN-\(\alpha\) pretreatments confer opposite effects depending on their durations}
\label{sec:orgd9b6c5f}

\begin{frame}[label={sec:orgdba9e2d}]{Fig 1A. HeLa Reporter Cell Line}
\url{\_20210422\_091025screenshot.png}
\end{frame}

\begin{frame}[label={sec:org78d5f86}]{Fig. 1D Schematic of IFN-\(\alpha\) pretreatment experiments}
\url{\_20210422\_104851screenshot.png}
\end{frame}

\begin{frame}[label={sec:orga011b1e}]{Fig. 1D A diagram of the microfluidic set-up}
\url{\_20210422\_105123screenshot.png}
\end{frame}

\begin{frame}[label={sec:orge8088cb}]{Fig. 1B Time Lapse of Cells treated for IFN-\(\alpha\) for 48 hours}
\url{\_20210422\_091431screenshot.png}
\end{frame}

\begin{frame}[label={sec:orgdec43c3}]{Fig. 1B Time traces of nuclear/cytoplasmic STAT1-mCherry and PIRF9-YFP signals of the cell}
\url{\_20210422\_091808screenshot.png}
\end{frame}

\begin{frame}[label={sec:org09960b9}]{Fig. 1C Averaged time traces of nuclear/cytoplasmic STAT1-mCherry, PIRF9-YFP}
\url{\_20210422\_100656screenshot.png}
\end{frame}


\begin{frame}[label={sec:org258833f}]{Fig. Supp 1D Time course western blots showing the dynamics of phosphorylation (pY701), IRF9 and expression of STAT1}
\url{\_20210422\_093026screenshot.png}
\end{frame}

\begin{frame}[label={sec:orgd22d88c}]{Fig. 1E PIRF9-driven YFP induction response to the second IFN-\(\alpha\) treatment}
\url{\_20210422\_102211screenshot.png}
\end{frame}



\section{USP18 is responsible for desensitization induced by the prolonged IFN-\(\alpha\) pretreatment}
\label{sec:orgfad5425}

\begin{frame}[label={sec:orgb79726b}]{Fig. 2A Time-lapse images of STAT1 nuclear translocation}
\url{\_20210422\_112910screenshot.png}
\end{frame}

\begin{frame}[label={sec:org50accf8}]{Fig. Supp. 2A Western blots of USP18 expression in WT and USP18-KD cells}
\url{\_20210422\_121123screenshot.png}
\end{frame}

\begin{frame}[label={sec:org715779a}]{Fig. 2B YFP induction response to the second IFN-\(\alpha\) treatment in USP18-KD}
\url{\_20210422\_111043screenshot.png}
\end{frame}

\begin{frame}[label={sec:org985da8a}]{Fig. 1F Amounts of PIRF9-YFP induction by the second IFN-\(\alpha\) stimulation}
\url{\_20210422\_111453screenshot.png}
\end{frame}

\begin{frame}[label={sec:org21a7a6e}]{Fig. 2C Amounts of PIRF9-YFP induction in USP18-KD cells by the second IFN-\(\alpha\) stimulation}
\url{\_20210422\_111412screenshot.png}
\end{frame}

\section{Computational modeling suggests a delayed negative feedback loop through USP18}
\label{sec:orge40d00a}
\begin{frame}[label={sec:org65d0844}]{Fig. 3A Simple kinetic model of the IFN-driven gene regulatory network}
\url{\_20210422\_113839screenshot.png}
\end{frame}

\begin{frame}[label={sec:org6a2c454}]{Fig. Supp 3A kinetic model of the IFN-driven gene regulatory network with parameters}
\url{\_20210422\_115407screenshot.png}
\end{frame}

\begin{frame}[label={sec:org1319d70}]{Fig. 3C Amounts of PIRF9-YFP induction by the second IFN-α stimulation Predicted by model simulations}
\url{\_20210422\_114424screenshot.png}
\end{frame}


\begin{frame}[label={sec:org1bebcec}]{Fig. Supplement 3B Model fitting results}
\url{\_20210422\_115253screenshot.png}
\end{frame}

\begin{frame}[label={sec:org46c32b4}]{Fig 3D. Experimental design with repetitive IFN pulses}
\url{\_20210422\_115744screenshot.png}
\end{frame}


\begin{frame}[label={sec:org2f895b8}]{Fig. 3E Model prediction of the responses to pulse versus sustained IFN inputs}
\url{\_20210422\_115809screenshot.png}
\end{frame}

\begin{frame}[label={sec:org78c48f6}]{Fig. 3F Experimental data of the responses to pulse versus sustained IFN inputs}
\url{\_20210422\_115854screenshot.png}
\end{frame}

\section{The kinetics of USP18 upregulation by IFN is heterogeneous in single cells}
\label{sec:orgaeb453c}
\begin{frame}[label={sec:orgea592ff}]{Fig. 4A Dual reporter cell line schematic}
\url{\_20210422\_120826screenshot.png}
\end{frame}

\begin{frame}[label={sec:org4cd7d9c}]{Fig. 4B Time traces of PIRF9-YFP and PUSP18-CFP of a single cell in response to IFN-α}
\url{\_20210422\_120948screenshot.png}
\end{frame}

\begin{frame}[label={sec:org922249b}]{Fig 4C. Distributions of PIRF9 and PUSP18 activation times in single cells}
\url{\_20210422\_121313screenshot.png}
\end{frame}

\begin{frame}[label={sec:org0d51d52}]{Fig 4D. Distributions of delay times in single cells}
\url{\_20210422\_121358screenshot.png}
\end{frame}

\begin{frame}[label={sec:orgd564583}]{Fig 4E. Representative time traces of PIRF9 and PUSP18 in a single cell from each group}
\url{\_20210422\_121444screenshot.png}
\end{frame}

\section{Cell cycle phases differentially regulate USP18 expression}
\label{sec:orgb087d79}

\begin{frame}[label={sec:orgd25fd49}]{Fig 4E. Delay times as a function of the percentages of cell cycle progression upon IFN treatment onset}
\url{\_20210422\_122543screenshot.png}
\end{frame}


\begin{frame}[label={sec:org02ab943}]{Fig 5A. Delay times in cells treated with different cell cycle perturbation}
\url{\_20210422\_121847screenshot.png}
\end{frame}



\begin{frame}[label={sec:org5cb66c2}]{Fig 5B. CDK2 activity reporter Schematic}
\url{\_20210422\_122704screenshot.png}
\end{frame}

\begin{frame}[label={sec:org03944c5}]{Fig 5C. Nuclear DHB and PUSP18-driven gene expression}
\url{\_20210422\_122941screenshot.png}
\end{frame}

\begin{frame}[label={sec:org36d29a0}]{Fig 5D. Effect of decitabine on DNA methylation and nucleosome occupancy}
\url{\_20210422\_123028screenshot.png}
\end{frame}

\begin{frame}[label={sec:org5241cac}]{Fig 5D. Distribution of delay times upon decitabine treatment}
\url{\_20210422\_123104screenshot.png}
\end{frame}

\section{Cell-cycle-gated feedback control shapes single-cell responses to repetitive IFN inputs}
\label{sec:org906da3c}

\begin{frame}[label={sec:org81cb677}]{Fig. 6A simple model of the IFN-driven gene regulatory network}
\url{\_20210422\_123240screenshot.png}
\end{frame}

\begin{frame}[label={sec:orgc5cb797}]{Fig. 6C cell-cycle gated feedback control simulated responses under different pretreatment conditions}
\url{\_20210422\_123535screenshot.png}
\end{frame}

\begin{frame}[label={sec:org7ce52a5}]{Fig. 6D cell-cycle gated feedback control experimental responses under different pretreatment conditions}
\url{\_20210422\_123547screenshot.png}


\begin{quote}
Higher levels of USP18 expression by the prolonged pretreatment lead to reduced IRF9 induction upon the second stimulation at the single-cell level, qualitatively in agreement with our experimental data
\end{quote}
\end{frame}


\section{Conclusion}
\label{sec:orgd3dd864}

\begin{frame}[label={sec:org7b2ea05}]{Conclusion}
\begin{itemize}
\item The effects of IFN pretreatments depend on their input durations
\item The G1 and early S phases enable an open window for immediate USP18
upregulation upon the IFN treatment
\begin{itemize}
\item If they miss the window the USP18 induction has to wait for G1 of the next cell cycle
\end{itemize}
\item SARS-CoV-2 is especially sensitive to type I IFNs
\begin{itemize}
\item IFN pretreatment a potential strategy to prevent SARS-CoV-2 infection
\end{itemize}
\end{itemize}
\end{frame}
\end{document}
