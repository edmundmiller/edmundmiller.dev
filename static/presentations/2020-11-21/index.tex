% Created 2020-11-20 Fri 14:05
% Intended LaTeX compiler: pdflatex
\documentclass[bigger]{beamer}
\usepackage[utf8]{inputenc}
\usepackage[T1]{fontenc}
\usepackage{graphicx}
\usepackage{grffile}
\usepackage{longtable}
\usepackage{wrapfig}
\usepackage{rotating}
\usepackage[normalem]{ulem}
\usepackage{amsmath}
\usepackage{textcomp}
\usepackage{amssymb}
\usepackage{capt-of}
\usepackage{hyperref}
\usetheme[progressbar=foot]{metropolis}
\author{Edmund Miller}
\date{2020-11-21 Sat}
\title{The Pentose Phosphate Pathway}
\hypersetup{
 pdfauthor={Edmund Miller},
 pdftitle={The Pentose Phosphate Pathway},
 pdfkeywords={},
 pdfsubject={},
 pdfcreator={Emacs 28.0.50 (Org mode 9.5)}, 
 pdflang={English}}
\begin{document}

\maketitle


\section{Overview of the Pentose Phosphate Pathway}
\label{sec:orgebbb4b8}

\begin{frame}[label={sec:org4437bee}]{Overview}
\begin{center}
\includegraphics[height=0.7\linewidth]{/home/emiller/sync/org/roam/data/ab/3324f3-9dfb-431a-ae5b-3f64553d4baf/_20201120_140305Pentose_Phosphate_Pathway.png}
\end{center}
\end{frame}

\section{Oxidative phase}
\label{sec:orgfd95291}


\begin{frame}[label={sec:orgcd26751}]{Oxidative phase}
\begin{itemize}
\item The breakdown of glucose, in glycolysis provides the first 6-carbon,
glucose-6-phosphate
\item The steps are \alert{irreversible}
\end{itemize}

Glucose-6-phosphate + 2 NADP+ ->
\begin{block}{Noteable Products}
Ribulose-5-phosphate + 2 NADPH
\end{block}
\end{frame}

\begin{frame}[label={sec:org74ed4f3}]{Step 1}
\begin{itemize}
\item Glucose-6-phosphate is oxidized
\item NADPH is produced as a byproduct as NADP+ is reduced
\item 6-phosphogluconate is formed
\end{itemize}
\end{frame}

\begin{frame}[label={sec:org67d9f52}]{Step 2}
\begin{itemize}
\item Oxidative decarboxylation of 6-phosphogluconate
\begin{itemize}
\item CO\textsubscript{2} is released
\item The eletrons released are used to reduce NADP+ to NADPH
\end{itemize}
\item Ribose-5-phosphate is produced
\end{itemize}
\end{frame}

\section{Non-oxidative phase}
\label{sec:org6eb6c99}

\begin{frame}[label={sec:org6f4f688}]{Non-oxidative phase}
\begin{itemize}
\item The steps are \alert{reversible}
\item Allows different molecules to enter the pathway at different stages
(interconvert sugars)
\end{itemize}
\begin{block}{Noteable Products}
\begin{itemize}
\item Ribose-5-Phosphate
\item 2 fructose-6-phosphate
\item glyceraldehyde-3-phosphate
\end{itemize}
\end{block}
\end{frame}

\begin{frame}[label={sec:org13fb0ca}]{Step 3 - Rearrangement of Ribulose-5-phosphate}
\begin{block}{Ribose-5-phosphate}
\begin{itemize}
\item isomerization (exchange of groups between carbons)
\end{itemize}
\end{block}
\begin{block}{Xylulose-5-phosphate}
\begin{itemize}
\item epimerization (exchange of groups on a single carbon)
\end{itemize}
\end{block}

\begin{block}{The three pentose phosphates are in equilibrium because the reactions are reversible}
\end{block}
\end{frame}

\section{Study Guide Review}
\label{sec:orgf2b6bec}
\begin{frame}[label={sec:org794cd8c}]{Understand that glucose can be used as a source of NADPH \alert{and} of building blocks for biosynthetic pathways}
\begin{itemize}
\item Glucose is broken down in glycolysis into Glucose-6-phosphate
\item Glucose-6-phosphate is passed to the \alert{oxidative phase}
\item The \alert{oxidative phase} and produces \alert{2 NADPH}

\item The \alert{non-oxidative phase} can produce \alert{ribose-5-phosphate}
\begin{itemize}
\item \alert{ribose-5-phosphate} is used for nucleotide biosynthesis
\end{itemize}
\end{itemize}
\end{frame}

\begin{frame}[label={sec:orgad8fb33}]{Two oxidative reactions of the pathway provide NADPH for biosynthesis}
\begin{itemize}
\item The oxidative phase and produces 2 NADPH one for each oxidation
\end{itemize}
\end{frame}

\begin{frame}[label={sec:orgff8d13e}]{The non-oxidative reactions of the pathway provide ribose-5-phosphate for the biosynthesis of nucleotides}
\begin{itemize}
\item Rearrangement of \alert{Ribulose-5-phosphate} 
\begin{itemize}
\item Produces \alert{Ribose-5-phosphate} 
\begin{itemize}
\item isomerization (exchange of groups between carbons)
\end{itemize}
\end{itemize}
\end{itemize}
\end{frame}

\begin{frame}[label={sec:org61ce90b}]{The products of the non-oxidative branch (fructose-6-phosphate and glyceraldehyde-3-phosphate) can be returned to glycolysis or gluconeogenesis}
\begin{block}{When \alert{more} ribose-5-P than NADPH is required}
\begin{itemize}
\item Fructose-6-P and glyceraldehyde-3-P from Glycolysis are fed into the
non-oxidative branch
\begin{itemize}
\item The reaction then runs in \uline{reverse} to make ribose-5-P with \alert{no} NADPH generated
\end{itemize}
\item \alert{No} carbon is returned to glycolysis
\end{itemize}
\end{block}
\end{frame}

\begin{frame}[label={sec:orgc814f0b}]{The pentose phosphate pathway can operate in four different modes according to the cell’s requirements for NADPH, ribose-5-phosphate and ATP}
\begin{block}{When \alert{both} ribose-5-P and NADPH are required}
\end{block}
\begin{block}{When \alert{more} ribose-5-P than NADPH is required}
\end{block}
\begin{block}{When \alert{more} NADPH than ribose-5-P is required}
\end{block}
\begin{block}{When \alert{both} NADPH and ATP are needed, but ribose-5-P is not}
\end{block}
\end{frame}
\begin{frame}[label={sec:org1e35aa7}]{When \alert{both} ribose-5-P and NADPH are required}
\begin{itemize}
\item The predominant mode is to make NADPH and to make ribose-5-P
\item The oxidative reactions predominate
\item no carbon is returned to Glycolysis
\end{itemize}
\end{frame}

\begin{frame}[label={sec:org9769b09}]{When \alert{more} ribose-5-P than NADPH is required}
\begin{itemize}
\item Fructose-6-P and glyceraldehyde-3-P from Glycolysis are fed into the
non-oxidative branch
\begin{itemize}
\item The reaction then runs in \uline{reverse} to make ribose-5-P with \alert{no} NADPH generated
\end{itemize}
\item \alert{No} carbon is returned to glycolysis
\end{itemize}
\end{frame}

\begin{frame}[label={sec:orge66720f}]{When \alert{more} NADPH than ribose-5-P is required}
6 Glucose-6-phosphate -> 6 ribose-5-P + 12 NADPH + 6 CO\textsubscript{2} by the pentose phosphate pathway 

6 ribose-5-P -> 4 fructose-6-P + 2 glyceraldehyde-3-P

4 fructose-6-P + 2 glyceraldehyde-3-P -> 5-glucose-6-P by gluconeogenesis

Net reaction:

Glucose-6-phosphate + 12 NADP+ -> 6 CO\textsubscript{2} + 12 NADPH
\end{frame}

\begin{frame}[label={sec:org8f4280b},fragile]{When \alert{both} NADPH and ATP are needed, but ribose-5-P is not}
 \begin{itemize}
\item The same as the previous case, but the \texttt{fructose-6-P} and \texttt{glyceraldehyde-3-P}
are fed into glycolysis to generate ATP
\end{itemize}
\end{frame}
\end{document}